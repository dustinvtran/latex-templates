%##############################################################################
% Preamble
%##############################################################################

%##############################################################################
% Document Class, Font/Font Size, & Packages
%##############################################################################

\documentclass[11pt]{article}
\usepackage[utf8x]{inputenc}                    %Support for some unicode (e.g. emdash) without using XeTeX.

%##############################################################################
% Font, Margins, & Spacing
%##############################################################################

\renewcommand{\rmdefault}{ptm}
\usepackage[hmargin=1in,vmargin=1in]{geometry}
\parindent0em
\parskip1em

%##############################################################################
% Title & Headings
%##############################################################################

\title{\vspace{-6.5ex}\sc{\large Statement of Purpose}\\[-6.0ex]}
%\author{\sc{\large
%        Dustin Tran}\\[0.25em]
%        }
\date{}
\pagestyle{empty}

\begin{document}
\maketitle
\thispagestyle{empty}

%##############################################################################
% Begin Document
%##############################################################################

% Prompt:
%The Statement of Purpose should describe succinctly your reasons for applying to the proposed program at Stanford, your preparation for this field of study, research interests, future career plans, and other aspects of your background and interests which may aid the admissions committee in evaluating your aptitude and motivation for graduate study. The Statement of Purpose should not exceed two pages.

% Notes:
%* Directly ripped from mit_or_sop.tex.

Having pursued research in several fields of algebra and geometry, I have developed the most interest in modelling high dimensional complex systems, and I am fascinated by stochastic optimization methods which do so. I aim to pursue the M.S. degree in Management Science \& Engineering in order to further my studies onto such subjects, and to tackle more concentrated problems with continued research. Ultimately, my professional goal is to attain a research-based position in academia or the technology industry. As such, I strongly believe that this program will offer me a comprehensive understanding of the subjects as well as a thorough research experience that will allow me to do so.

In my second year of university, I began undertaking intensive graduate-level courseloads, and have for the past two years experienced first-hand what it is like to be a graduate student in algebra and geometry. The wide array of graduate courses I've completed at Berkeley ranges from ordinary differential equations, differential geometry, Riemannian geometry, to even higher level abstractions such as algebraic topology, algebraic geometry, and algebraic number theory. I learned side-by-side with just the handful of graduate students in every course, and have cultivated the same practices they do both inside and outside the classroom: tackle highly intricate problems with a considerable degree of independent research, read and learn extensively from supplementary resources, attend and talk at seminars, discuss material with colleagues, and frequently converse with professors during lectures and appointment-set office hours.

I've also ensured to supplement my study by tackling current problems independently with several professors. For example, my initial passion was in low dimensional topology and its modelling applications, and I completed research in this subject under Prof. Robion Kirby. As I learned algebraic topology and complex analysis concurrently within the semester, I was able to expand upon my knowledge of homology and surgery theory by applying them to examples from other scientific disciplines. One of the experiences I am most fond of was reforming well-established models such as Black-Scholes in mathematical finance--a field I previously knew nothing about--and applying techniques in topology such as persistent homology. Doing so offers more effective qualitative analysis of data than one could otherwise.

Since then I also collaborated with Prof. Maciej Zworski and analyzed Riemann surfaces in pricing under stochastic volatility models, including the 3/2, Heston, and SABR. Painting these figures under more abstract topological contexts provides inquiry into its latent geometric structure, and this especially has been fascinating to see the crossovers with manifold learning. I eventually produced a research paper regarding the more generic theory of Stein manifolds, and I combined knowledge from my research in symplectic geometry with Prof. Michael Hutchings in order to write another one about exotic symplectic structures, a subject which has seen much recent development.

Above all my interests have steadily gravitated toward quantitative applications, notably computing and optimization in modelling. I first pursued programming as a hobby back in my first year of university, and it has continued to take a larger presence in my research. For example, because of their high dimensionality, I often solved these modelling problems with Monte Carlo methods. I placed much emphasis into optimizing computing time and storage by following practices in parallel processing and current algorithms, e.g., with sparse solvers. Most recently I have been studying randomized algorithms for low rank approximations in numerical linear algebra, and programming an article's detailed steps has been invaluable to understanding how they work.

I've also accumulated experience with programming and optimization in my various teaching assistantships. As an undergraduate instructor, I've frequently contributed to the course material and implemented several demonstrations in MATLAB and Python. In order to garner interest in the subjects the students are learning, I often provide motivating examples, such as an animation which compares the performance of iterative and direct solvers to approximate a 3D surface.

As my background has primarily been algebraic and geometric driven, I would like to continue to learn techniques and to address challenges that are oriented toward more quantitative methods: modelling and stochastic optimization. The research project component of your program and comprehensive subject offerings allow me the possibility to not only accumulate such knowledge but to directly confront and solve these problems at a more intimate level. Furthermore, I believe that my past experiences indicate that I have an excellent understanding of many interrelated subjects, and most of all that I have the ability to further advance my knowledge of these fields.

%##############################################################################
% End Document
%##############################################################################

\end{document}
